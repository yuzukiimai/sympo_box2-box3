\documentclass[twocolumn, 9pt]{jsproceedings}
\RequirePackage[l2tabu, orthodox]{nag}  % 古いコマンドやパッケージを使用した場合に警告する
\usepackage[all, warning]{onlyamsmath}  % amsmath が提供しない数式環境を使用した場合に警告する
\usepackage{flushend}  % 最終ページの2カラムの左右の高さを揃える



\usepackage{otf}
\usepackage[ipa]{pxchfon}
\usepackage{caption}
\usepackage{graphicx}

% タイトル
\title{つくばチャレンジ 2023 における\\千葉工業大学未来ロボティクス学科 box2,box3チームの取り組み}

\author{○今井 悠月,井口 颯人,樋高 聖人,春山 健太,藤原 柾,\CID{8705}橋 祐樹,白須 和暉,野村 駿斗,
望月 悠矢,\\馬場 琉生,村林 孝太郎,桜井 真希,中村 雄一,長島 昂生,\\上田 隆一,林原 靖男(千葉工大)}

\etitle{The activities of the Advanced Robotics Department box2 and box3 team of Chiba Institute of Technology in the Tsukuba Challenge 2023}

\eauthor{Yuzuki IMAI,  Hayato IGUCHI, Masato HIDAKA, Kenta HARUYAMA, Masaki FUJIWARA, \\Yuki TAKAHASHI, 
Kazuki SHIRASU, Hayato NOMURA, Yuya MOCHIZUKI, Ryusei BABA, \\Koutarou MURABAYASHI, Maki SAKURAI, 
Yuichi NAKAMURA, Kousei NAGASHIMA, \\Ryuichi UEDA and Yasuo HAYASHIBARA (CIT)}

\affiliation{千葉工業大学未来ロボティクス学科 box2, box3チーム}

\abstract{
  In this paper, we present the activities of the Advanced Robotics Department box2 team of 
  Chiba Institute of Technology in the Tsukuba Challenge 2023. 
  We developed autonomous outdoor mobile robots, and we tackled several challenges. 
  For example, We developed robots using machine learning and robots that can run in the rain.
}


\begin{document}
\maketitle

% 本文
\section{はじめに}
我々は,屋外でも正確に自律移動可能なロボットを目指し,その研究および開発の一環として
つくばチャレンジに参加している.
これまで本研究室\footnote{千葉工業大学未来ロボティクス学科 林原研究室}では,地図生成や自己位置推定を
中心に研究開発を行ってきた.近年では,防水機能や高い拡張性を有したオープンプラットフォームのロボットの
開発も行っている.つくばチャレンジ2016, 2017において,

本稿では,つくばチャレンジ2023に向けて取り組んだ内容に関して紹介する.

\section{ロボットの概要}
本研究室には,開発を続けている 3台のロボット ORNE-box,ORNE-box2,ORNE-box3 がある.
これらのロボットは,防水を意識した設計となっており,屋外環境での使用を想定した設計となっている.
さらにオープンプラットフォームのロボットとなっており,いずれも公開している.\\
\subsection{ハードウェア}
次に,ORNE-box-Series の外観を,ハードウェア構成をに示す.
これらのロボットは,i-Cart middle をベースとしている.\\

\begin{figure}[h]
  \centering
  \begin{minipage}[b]{0.3\linewidth}
    \centering
    \includegraphics[width=30mm]{fig/alpha.pdf}
    \caption*{(a) ORNE-α}
  \end{minipage} 
  \hspace{0.03\columnwidth}
  \begin{minipage}[b]{0.3\linewidth}
    \centering
    \includegraphics[height=34mm]{fig/box.pdf}
    \caption*{(b) ORNE-box}
  \end{minipage}
  \begin{minipage}[b]{0.3\linewidth}
    \centering
    \includegraphics[height=34mm]{fig/box2.pdf}
    \caption*{(c) ORNE-box2}
  \end{minipage}
  \caption{ORNE-Series}
  \label{fig:orne-series}%\vspace*{-2mm}
\end{figure}


\begin{table}[h]
  \centering
    \caption{Specifications of the robots}
    \scalebox{0.8}{
    \begin{tabular}{|l||c|c|c|}  \hline
       & ORNE-α & ORNE-box & ORNE-box2 \\ \hline \hline
      Depth[mm] & 690 & 106,800円 & A14 Bionic \\ \hline
      Wide[mm] & 560 & 85,800円 & A14 Bionic \\ \hline
      Height[mm] & 770 & 74,800円 & A14 Bionic \\ \hline
      Wheel diameter[mm] & \multicolumn{3}{|c|}{304}\\ \hline
      Battery & \multicolumn{3}{|c|}{LONG WP12-12}\\ \hline
      Motor & \multicolumn{3}{|c|}{Oriental motor TF-M30-24-3500-G15L/R}\\ \hline
      Driving system & \multicolumn{3}{|c|}{Power wheeled steering}\\ \hline
      2D-LiDAR & URM-40LC-EW & None & UTM-30LX-EW\\
      & (HOKUYO) & & (HOKUYO)\\ \hline
      3D-LiDAR & None & R-fans-16 & VLP-16\\
      & & (SureStar) & (Velodyne)\\ \hline
      IMU & ADIS16465 & \multicolumn{2}{|c|}{ADIS16475}\\ 
      & (Analog devices) & \multicolumn{2}{|c|}{Analog devices}\\ \hline
      GNSS receiver & None & \multicolumn{2}{|c|}{u-blox SCR-u2t}\\ \hline
      Camera & CMS-V43BK & \multicolumn{2}{|c|}{None}\\ 
      & (Sanwa supply) & \multicolumn{2}{|c|}{}\\ \hline
    \end{tabular}
  }
  \end{table}


\subsection{ソフトウェア}
本チームでは, 従来よりROS(Robot Operating System)のnavigation stack[1]をもとに開発されたシステム
であるorne\_navigationにより自律走行させている. 
\figref{fig:soft-fig}に開発しているロボットのソフトウェアを含むシステム構成を示す. 
このシステムは, 2D-LiDARを用いたMonte Carlo Localization(MCL)により確率的に自己位置を推定し, 
経路計画に基づいて自律走行している. また, GitHubのopen-rdc[2]でプロクラムを公開している.

\begin{figure}[h]
  \centering
  \includegraphics[width=85mm]{fig/software.pdf}
  \caption{Structure of the system.}
  \label{fig:soft-fig}%\vspace*{-2mm}
\end{figure}

\section{おわりに}
本稿では,千葉工業大学未来ロボティクス学科 box2チームで開発しているロボットの概要とシステムの構成に
関して述べた.また,つくばチャレンジ2023に向けた取り組みについて紹介した.

\section*{謝辞}
つくばチャレンジ実行委員会の皆様およびつくば市の皆様に感謝申し上げます.また,上田研究室の皆様には
つくばチャレンジ2023の参加にあたり,ご意見やご協力をいただき感謝申し上げます.


% 参考文献
% \small
\footnotesize
\begin{thebibliography}{99}
\bibitem{LaTeX-Cmd}
\LaTeX~コマンド集.\\
\url{http://www.latex-cmd.com/}

\bibitem{TeX-Wiki}

\end{thebibliography}
\normalsize

\end{document}
